\documentclass{article}
\usepackage{enumitem,amssymb}
\usepackage{lipsum} % for filler text
\usepackage{geometry}
\geometry{
	paperwidth=5.5in,
	paperheight=8.5in,
	top=0.25in,
	bottom=0.5in,
	left=0.25in,
	right=0.25in,
	footskip=20pt  % Reserve space for footer
}
\usepackage{qrcode}
\newlist{todolist}{itemize}{2}
\setlist[todolist]{label=$\square$}
\usepackage{easylist}
\usepackage{fancyhdr}
\pagestyle{fancy}
\fancyfoot{} % clear all footer fields
\fancyfoot[LE,RO]{\thepage}           % page number in "outer" position of footer line
\fancyfoot[RE,LO]{How To Build A Common vVERSIONNUMBER} % other info in "inner" position of footer line
\usepackage{hyperref}
\usepackage{tikz}

\newcommand{\linedpage}{%
	\noindent
	\foreach \n in {1,...,10}{%
		\rule{\textwidth}{0.4pt}\\[\baselineskip]
	}
}

\usepackage{multicol}

\begin{document}
	
{\huge \textbf{Building A Common}}
\pagebreak

\vspace{1cm}

{\huge \textbf{Referenced Readings}}

\begin{multicols}{2}
\begin{enumerate}
	
	\item \textbf{ Governing the Commons  by Ellinor Ostrum (Specifically p90-p102) } 
	
	\qrcode[height=30mm]{https://archive.org/details/governingthecommons}

	\item \textbf{The Democracy Project by David Graeber  (Specifically Chapter 2)  "Consensus"} 
	
	\qrcode[height=30mm]{https://theanarchistlibrary.org/library/david-graeber-the-democracy-project#toc21}
	
	\item \textbf{Building Belonging by Yana Ludwig} 
	
	\qrcode[height=30mm]{https://github.com/How-To-Build-a-Commons/Scope-Sequencing/releases/download/0.0.11/Building-Belonging-FINAL-E-book-iwmkwg.epub}
	
	\item \textbf{Debt: The First 5000 Years by David Graeber  (Chapter 11)} 
	
	\qrcode[height=30mm]{https://files.libcom.org/files/__Debt__The_First_5_000_Years.pdf}
	
	\item \textbf{"Violence Vouchers: a descriptive account of property" by Matt Bruenig} 
	
	\qrcode[height=30mm]{https://mattbruenig.com/2014/03/28/violence-vouchers-a-descriptive-account-of-property/}
	
	\item \textbf{Source Of This Document}
	
	\qrcode[height=30mm]{https://github.com/How-To-Build-a-Commons/Scope-Sequencing/releases/tag/VERSIONNUMBER}
	
	 

	
\end{enumerate}
\end{multicols}

\pagebreak
{\huge \textbf{What is a Common?}}

\textbf{Governing the Commons by Ellinor Ostrum Table 3.1}


Design principles illustrated by long-enduring CPR institutions

\textbf{1. Clearly defined boundaries}

Individuals or households who have rights co withdraw resource units from the CPR must be clearly defined, as must the boundaries of the CPR itself.

\textbf{2. Congruence between appropriation and provision rules and local conditions}

 Appropriation rules restricting time, place, technology, and/or quantity of resource units are related to local conditions and co provision rules requiring labor, material, and/or money.

\textbf{3. Collective-choice arrangements}

Most individuals affected by the operational rules can participate in modifying the operational rules.

\textbf{4. Monitoring}

Monitors, who actively audit CPR conditions and appropriator behavior, are accountable to the appropriators or are the appropriators.

\textbf{5. Graduated sanctions}

Appropriators who violate operational rules arc likely to be assessed graduated sanctions (depending on the seriousness and context of the offense) by other appropriators, by officials accountable to these appropriators, or by both.

\textbf{6. Conflict-resolution mechanisms}

Appropriators and their officials have rapid access to low-cost local arenas to resolve conflicts among appropriators or between appropriators and officials.

\textbf{7. Minimal recognition of rights to organize}

The rights of appropriators to devise their own institutions arc not challenged by external governmental authorities.

For CPRs that are parts of larger systems:

\textbf{8. Nested enterprises}

Appropriation, provision, monitoring, enforcement, conflict resolution, and governance activities are organized in multiple layers of nested enterprises.





\pagebreak
{\huge \textbf{What is Consensus Process?}}

Consensus is a process where everyone should be able to weigh in equally on a decision, and no one should be bound by a decision they detest."
this boils doing in practice to: Everyone who feels they have something relevant to say about a proposal ought to have their perspectives carefully considered.

Everyone who has strong concerns or objections should have those concerns or objections taken into account and, if possible, addressed in the final form of the proposal.

Anyone who feels a proposal violates a fundamental principle shared by the group should have the opportunity to veto (“block”) that proposal.

No one should be forced to go along with a decision to which they did not assent.

1) someone makes a proposal for a certain course of action

2) the facilitator asks for clarifying questions to make sure everyone understands precisely what is being proposed

3) the facilitator asks for concerns

3.1)during the discussion those with concerns may suggest friendly amendments to the proposal to address the concern, which the person originally bringing the proposal may or may not adopt

3.2)there may or may not be a temperature check about the proposal, an amendment, or the seriousness of a concern

3.3)in the course of this the proposal might be scotched, reformulated, combined with other proposals, broken into pieces, or tabled for later discussion.

4) the facilitator checks for consensus by:

4.1) asking if there are any stand-asides. By standing aside one is saying “I don’t like this idea, and wouldn’t take part in the action, but I’m not willing to stop others from doing so”. It is always important to allow all those who stand aside to have a chance to explain why they are doing so.

4.2) asking if there are any blocks. A block is not a “no” vote. It is much more like a veto. Perhaps the best way to think of it is that it allows anyone in the group to temporarily don the robes of a Supreme Court justice and strike down a piece of legislation they consider unconstitutional; or, in this casein violation of the fundamental principles of unity or purpose of being of the group.{42},
Footnote {42} I should note that the usual language in Occupy Wall Street is that a block has to be based on a “moral, ethical, or safety concern that’s so strong you’d consider leaving the movement were the proposal to go forward”.

\pagebreak
{\huge \textbf{Why Build a Common?}}

From the Yana Ludwig reading:

\begin{enumerate}
	\item Spiritual or Religious
	\item 	Cultural Preservation
	\item 	Social Experimentation
	\item 	Service-based
	\item 	Economic Security
	\item 	Identity-based Safe Havens
	\item 	Lifestyle and Comfort Enhancement
	\item 	Ecological Sustainability
\end{enumerate}


\pagebreak
{\huge \textbf{What we Owe to Each Other}}


"In a typical village, the only people likely to pay cash were passing
travelers, and those considered riff-raff: paupers and ne'er-do-wells so
notoriously down on their luck that no one would extend credit to
them. Since everyone was involved in selling something, however j ust
about everyone was both creditor and debtor; most family income took
the form of promises from other families; everyone knew and kept
count of what their neighbors owed one another; and every six months
or year or so, communities would hold a general public " reckoning,"
cancelling debts out against each other in a great circle, with only those
differences then remaining when all was done being settled by use of
coin or goods." (Debt: The First 5000 Years Page 327)



\large \textbf{Maker Checks}

A modern version of the village exchange loops would be the idea of Maker Checks. 




\pagebreak
{\huge \textbf{What is Property?}}
{\fontsize{9pt}{9pt}\selectfont
\large Legal Definitions:

This category is refered to as "Private Property" and is a collection of a few distinct rights. These are the rights enforced by and recognized by law. 

\begin{enumerate}
	
	\item \textbf{Usefruct} - the right use use and receive the value from a piece of the physical world

	\item \textbf{Destruction} - the right to destroy a piece of the physical world

	\item \textbf{Exclusion} -  the right to exclude others
 
	\item \textbf{Increase (rent)} - the right to receive rents
\end{enumerate}
	

\large \textbf{Categories in a Commons}

Within a Commons, once the external entity has been assigned Private Property rights within the law, then that "bubble" can ascribe the rights according to its own rules. The commons can allocate areas of the pieces of the physical world that it manages (the clearly defined boundaries). The allocations can be for consumed aspects (the increase), it can be to assign use terms (Usefruct), and it can determine to what degree exclusion and destruction are used within the commons.

\large \textbf{Personal Property}: The items or space that is exclusively assigned along the legal definition. These can be either private property that came with someone into the commons, or can be the appropriators allocated share of some bounty. 

 \large \textbf{Commons Space}: This is pieces of the physical world that may have Usefruct allocated to members, or the public. These generally have the right of destruction held withing the commons (for repairs and upgrades). The right to exclude is also held by the commons itself to determine if and when people can or will be excluded. 
 
  \large \textbf{Public Space}: This is space held by the state that assigned property rights. Examples would be the roads, infrastructure and public lands such as parks. 
}

\pagebreak

\huge \textbf{HOWTO Build Common Workshop}

\linedpage

\pagebreak

Workshop 2

\linedpage

\pagebreak

Workshop 3

\linedpage

\vspace{1cm}

\pagebreak

Workshop 4

\linedpage

\vspace{1cm}

\pagebreak

\huge \textbf{A Call To Action}

{\fontsize{11pt}{11pt}\selectfont
I don't want to run a business. I don't want to manage or profit from
anyone. I should like to help everyone - if possible - Man, Woman -
young - old. We all want to help one another. Human beings are like
that. We want to live by each other's happiness - not by each other's
misery. We don't want to hate and despise one another. In this world
there is room for everyone. And the good earth is rich and can provide
for everyone. The way of life can be free and beautiful... but we have
lost the way.

Greed has poisoned our souls, has barricaded the world with hate, has
drone-striked us into misery and bloodshed. We have developed speed, but
we have shut ourselves in. The robotics that gives abundance has left us in
want. Our knowledge has made us cynical. Our cleverness, hard and
unkind. We think too much and feel too little. More than markets we need
humanity. More than profitability we need kindness and gentleness.
Without these qualities, life will be violent and all will be lost....

The internet and the smartphone have brought us closer together. The
very nature of these inventions cries out for the goodness in humanity -
cries out for universal kinship - for the unity of us all. Even now my
words could reach billions throughout the world - billions of
despairing men, women, and little children - victims of a system that
makes us torture and imprison each other in either the slavery of wage labor or the destitution of homelesness.

To those who can hear me, I say - do not despair. The misery that is now
upon us is but the passing of greed - the bitterness of those who fear
the way of human progress. The rise of automation has taken so much away
from so many, and the rewards, so far, given only to the few. As long as
\textbf{we} have the ability to invent, \textbf{what which was taken} can
\textbf{not} be kept for long....

To the Makers among us, don't give your \textbf{time,} your
\textbf{creativity,} your \emph{\textbf{life}} to a Corporation! An
organization designed for the sole purpose of greed; Designed to take as
much as possible for itself, to maximize shareholder value! Don't live
for greed, \textbf{Design for Freedom!} We the Makers have the power to
build Our Own automated production. \textbf{The power to create abundance!} We the Makers have the power to make this life Free and
Beautiful, to make this life a wonderful adventure! In the name of
Community and Free Information let us use this power, let us all unite
towards a common project. Let us design a free and open source
production system to provide \textbf{abundance for \emph{all}}\emph{.}
By the promise of abundance corporations have risen to power. But they
lie! They do not fulfill that promise. \textbf{They never will!} Free
markets free Corporations, but they enslave the people! Now let us
design a better world, a world of local sustainable automated abundance
- to do away with greed, with hate and domination. Let us design for a
world of reason, a world where science and progress will lead to all
humanity's freedom. \textbf{Makers!} in the name of Freedom and
Community, \textbf{let us all unite!}

(Credit: Charlie Chaplin, Reworked by Kevin Harrington)

}

\end{document}

\documentclass{article}
\usepackage{enumitem,amssymb}
\usepackage{lipsum} % for filler text
\usepackage{fancyhdr}
\pagestyle{fancy}
\usepackage{qrcode}
\newlist{todolist}{itemize}{2}
\setlist[todolist]{label=$\square$}
\usepackage{easylist}
\fancyfoot{} % clear all footer fields
\fancyfoot[LE,RO]{\thepage}           % page number in "outer" position of footer line
\fancyfoot[RE,LO]{Version VERSIONNUMBER} % other info in "inner" position of footer line
\usepackage{hyperref}

\usepackage{geometry}
\geometry{
	top=1in,
	bottom=1in,
	left=1in,
	right=1in
}
\usepackage{multicol}

\begin{document}
	
{\huge \textbf{How To Build A Commons}}


\vspace{1cm}

{\huge \textbf{Referenced Readings}}

\begin{multicols}{2}
\begin{enumerate}
	
	\item \textbf{ Governing the Commons  by Ellinor Ostrum (Specifically p90-p102) } 
	
	\qrcode{https://archive.org/details/governingthecommons}

	\item \textbf{The Democracy Project by David Graeber's  (Specifically Chapter 2)  "Consensus"} 
	
	\qrcode{https://theanarchistlibrary.org/library/david-graeber-the-democracy-project#toc21}
	
	\item \textbf{Building Belonging by Yana Ludwig} 
	
	\qrcode{https://github.com/How-To-Build-a-Commons/Scope-Sequencing/releases/download/0.0.11/Building-Belonging-FINAL-E-book-iwmkwg.epub}
	
	\item \textbf{Debt: The First 500 Years by David Graeber's  (Chapter 11)} 
	
	\qrcode{https://files.libcom.org/files/__Debt__The_First_5_000_Years.pdf}
	
	\item \textbf{Matt Bruenig 's "Violence Vouchers"} 
	
	\qrcode{	https://mattbruenig.com/2014/03/28/violence-vouchers-a-descriptive-account-of-property/}
	
	\item \textbf{Source Of This Document}
	
	\qrcode{https://github.com/How-To-Build-a-Commons/Scope-Sequencing/releases/tag/VERSIONNUMBER}
	
	 

	
\end{enumerate}
\end{multicols}

\pagebreak
{\huge \textbf{What is a Commons?}}

\textbf{Governing the Commons by Ellinor Ostrum Table 3.1}


Design principles illustrated by long-enduring CPR institutions

\textbf{1. Clearly defined boundaries}

Individuals or households who have rights co withdraw resource units from the CPR must be clearly defined, as must the boundaries of the CPR itself.

\textbf{2. Congruence between appropriation and provision rules and local conditions}

 Appropriation rules restricting time, place, technology, and/or quantity of resource units are related to local conditions and co provision rules requiring labor, material, and/or money.

\textbf{3. Collective-choice arrangements}

Most individuals affected by the operational rules can participate in modifying the operational rules.

\textbf{4. Monitoring}

Monitors, who actively audit CPR conditions and appropriator behavior, are accountable to the appropriators or are the appropriators.

\textbf{5. Graduated sanctions}

Appropriators who violate operational rules arc likely to be assessed graduated sanctions (depending on the seriousness and context of the offense) by other appropriators, by officials accountable to these appropriators, or by both.

\textbf{6. Conflict-resolution mechanisms}

Appropriators and their officials have rapid access to low-cost local arenas to resolve conflicts among appropriators or between appropriators and officials.

\textbf{7. Minimal recognition of rights to organize}

The rights of appropriators to devise their own institutions arc not challenged by external governmental authorities.

For CPRs that are parts of larger systems:

\textbf{8. Nested enterprises}

Appropriation, provision, monitoring, enforcement, conflict resolution, and governance activities are organized in multiple layers of nested enterprises.





\pagebreak
{\huge \textbf{What is a Consensus?}}

Consensus is a process where everyone should be able to weigh in equally on a decision, and no one should be bound by a decision they detest."
this boils doing in practice to: Everyone who feels they have something relevant to say about a proposal ought to have their perspectives carefully considered.

Everyone who has strong concerns or objections should have those concerns or objections taken into account and, if possible, addressed in the final form of the proposal.

Anyone who feels a proposal violates a fundamental principle shared by the group should have the opportunity to veto (“block”) that proposal.

No one should be forced to go along with a decision to which they did not assent.

1) someone makes a proposal for a certain course of action

2) the facilitator asks for clarifying questions to make sure everyone understands precisely what is being proposed

3) the facilitator asks for concerns

3.1)during the discussion those with concerns may suggest friendly amendments to the proposal to address the concern, which the person originally bringing the proposal may or may not adopt

3.2)there may or may not be a temperature check about the proposal, an amendment, or the seriousness of a concern

3.3)in the course of this the proposal might be scotched, reformulated, combined with other proposals, broken into pieces, or tabled for later discussion.

4) the facilitator checks for consensus by:

4.1) asking if there are any stand-asides. By standing aside one is saying “I don’t like this idea, and wouldn’t take part in the action, but I’m not willing to stop others from doing so”. It is always important to allow all those who stand aside to have a chance to explain why they are doing so.

4.2) asking if there are any blocks. A block is not a “no” vote. It is much more like a veto. Perhaps the best way to think of it is that it allows anyone in the group to temporarily don the robes of a Supreme Court justice and strike down a piece of legislation they consider unconstitutional; or, in this casein violation of the fundamental principles of unity or purpose of being of the group.{42},
Footnote {42} I should note that the usual language in Occupy Wall Street is that a block has to be based on a “moral, ethical, or safety concern that’s so strong you’d consider leaving the movement were the proposal to go forward”.

\pagebreak
{\huge \textbf{Why Build a Commons?}}

From the Yana Ludwig reading:

\begin{enumerate}
	\item Spiritual or Religious
	\item 	Cultural Preservation
	\item 	Social Experimentation
	\item 	Service-based
	\item 	Economic Security
	\item 	Identity-based Safe Havens
	\item 	Lifestyle and Comfort Enhancement
	\item 	Ecological Sustainability
\end{enumerate}


\pagebreak
{\huge \textbf{What we Owe to Each Other}}


"In a typical village, the only people likely to pay cash were passing
travelers, and those considered riff-raff: paupers and ne'er-do-wells so
notoriously down on their luck that no one would extend credit to
them. Since everyone was involved in selling something, however j ust
about everyone was both creditor and debtor; most family income took
the form of promises from other families; everyone knew and kept
count of what their neighbors owed one another; and every six months
or year or so, communities would hold a general public " reckoning,"
cancelling debts out against each other in a great circle, with only those
differences then remaining when all was done being settled by use of
coin or goods."

Debt: The First 500 Years Page 327

\large Maker Checks

A modern version of the village exchange loops would be the idea of Maker Checks. 




\pagebreak
{\huge \textbf{What is Property?}}

\large Legal Definitions:

\begin{enumerate}
	
	\item Usefruct - the right use use and receive the value from a piece of the physical world

	\item Destruction - the right to destroy a piece of the physical world

	\item Exclusion -  the right to exclude others
 
	\item Increase (rent) - the right to receive rents
\end{enumerate}
	
\large Philosophical definition: Violence vouchers

\large Holdings:

Saying someone owns a piece of the world obscures what is actually going on. Ownership is not a relationship between a person and a piece of the world. It is a relationship between a person and all other persons. It is a relationship that consists of the following threat: should someone else act upon this piece of the world, violence will be brought against them in order to cause them to desist.

When a state (or state-like entity) establishes a system of private property, all it really does is hand out violence vouchers to people who we call owners. 

\large Trading:

People do not trade pieces of the world. They trade violence vouchers. 

\large Rents:

People do not rent property from other people. They trade their violence voucher over some piece of the world in exchange for the person they are renting from agreeing to waive their right to redeem their violence voucher over some other piece of the world for some period of time.

A rent can thus be described as the acquisition of a violence voucher in exchange for temporarily waiving a right to redeem a violence voucher. A rent is when you leverage threats to redeem your violence vouchers in order to acquire violence vouchers from others without giving any violence vouchers in return.


\pagebreak
\huge HOWTO Build a Housing Cooperative

This is the workshop -- TODO

\vspace{1cm}
\end{document}

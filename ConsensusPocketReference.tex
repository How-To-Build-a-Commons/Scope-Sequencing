\documentclass{article}
\usepackage{geometry}
\geometry{
	paperwidth=2.75in,
	paperheight=4.25in,
	top=0.2in,
	bottom=0.5in,
	left=0.2in,
	right=0.2in
}

\usepackage{tikz}
\usetikzlibrary{shapes.geometric, arrows.meta, positioning}

% Define the flowchart command with scale and font size parameters
% Usage: \consensusflowchart{scale}{fontsize}
% Example: \consensusflowchart{0.8}{\small}
\newcommand{\consensusflowchart}[2]{%
	\begin{tikzpicture}[
	    node distance=1.5cm,
		scale=#1,
		transform shape,
		every node/.style={font=#2},
		xshift=-2cm,
		process/.style={rectangle, minimum width=3cm, minimum height=1cm, text centered, draw=black, fill=white, text width=3cm},
		decision/.style={diamond, minimum width=2cm, minimum height=1cm, text centered, draw=black, fill=gray!20, aspect=2},
		endpoint/.style={rectangle, minimum width=2.5cm, minimum height=1cm, text centered, draw=black, fill=white},
		block/.style={rectangle, minimum width=2.5cm, minimum height=1cm, text centered, draw=red, fill=white, text=red, line width=1pt},
		consensus/.style={rectangle, minimum width=2.5cm, minimum height=1cm, text centered, draw=green!60!black, fill=white, text=green!60!black, line width=1pt},
		arrow/.style={thick,->,>=Stealth}
		]
		% Nodes
		\node (discussion) [process] {Discussion};
		\node (proposal) [process, below=1.2cm of discussion] {Proposal};
		\node (test) [process, below=1.2cm of proposal] {Test for Consensus};
		\node (yes) [decision, right=1.2cm of test] {Yes};
		\node (no) [decision, left=1.2cm of test] {No};
		\node (modification) [process, below=1.2cm of test] {Modification to Proposal};
		\node (concerns) [endpoint, below=1.2cm of no] {Concerns Raised};
		\node (standAside) [endpoint, below=1.2cm of modification] {Stand Aside};
		\node (consensusAchieved) [consensus, below=1.2cm of yes] {Consensus Achieved};
		\node (blockNode) [block, below=1.2cm of concerns] {Block};
		\node (actionPoints) [endpoint, below=1.2cm of consensusAchieved] {Action Points};
		
		% Arrows
		\draw [arrow] (discussion) -- (proposal);
		\draw [arrow] (proposal) -- (test);
		\draw [arrow] (test) -- (yes);
		\draw [arrow] (test) -- (no);
		\draw [arrow] (yes) -- (consensusAchieved);
		\draw [arrow] (no) -- (concerns);
		\draw [arrow] (concerns) -- (modification);
		\draw [arrow] (concerns) -- (blockNode);
		\draw [arrow] (consensusAchieved) -- (actionPoints);
		\draw [arrow] (concerns) -- (standAside);
		\draw [arrow] (standAside) -- (consensusAchieved);
		
		% Curved arrows
		\draw [arrow, overlay] (no) to[out=120, in=180] (discussion);
		\draw [arrow, overlay] (modification) to[out=90, in=-90] (test);
		\draw [arrow, overlay] (blockNode) to[out=140, in=180] (discussion);
	\end{tikzpicture}%
}%


\begin{document}

	\begin{center}
		\consensusflowchart{0.6}{\footnotesize} 
	\end{center}
	 \newpage% Smaller version
	 
	\section{Consensus Process}
	
	\textbf{Discuss}: The item is discussed with the goal of identifying opinions and information on the topic at hand.
	
	\textbf{Proposal}: Based on the discussion a decision proposal is presented
	
	\textbf{Consensus Test}: The facilitator calls for consensus on the proposal. Each member must actively state whether they agree/consent, stand aside, or object,
	
	\textbf{Stand Aside}: Member does not support a proposal, but does not block
	
	\textbf{Block}: Member blocks on moral or ethical grounds and the proposal fails
	
	\newpage
	Page\par \#3\newpage
		\section{Roles}
	Roles: Essential for      Efficiency
	
	Essential Roles:        
	
	Facilitator: Keep the group on topic on time and within the rules        
	
	Timekeeper: Make sure no one rambles and keep on schedule         
	
	 Note Taker: Takes Notes     
	 
	 DLC Roles:   
	 
	 Empath: keep the emotional climate rational diffuse potential emotional conflicts    
	 
	 Devils Advocate: you know this one    Greeter: greet newcomers, inform them of what's happened
	
	\newpage
	\section{Things go right when...}
	\begin{enumerate}
	\item meeting more frequently
	\item use direct action tactics- paid staff are avoided- networked with other  consensus-based groups
	\item members monitor their own/others’ domineering behavior
	\item members reflected collectively on the distribution of power
	\item as many decisions as possible are left up to each person
	\item all decisions are treated as provisional
	\end{enumerate}
	\newpage
		\section{Things Go Wrong When...}
	\begin{enumerate}
	\item one voice is heard more than others
	\item complacency in a few leading decisions
	\end{enumerate}
	\newpage
	Back\par Cover \thispagestyle{empty} \newpage
	
	Intro to Achieving       Consensus
	
	Consensus Decision Making (CDM) 
	
	Intro and Reference
	
	"The commons are those things     that we all own together, that are neither privately owned,nor managed by the government"
	
	 \thispagestyle{empty} \newpage
\end{document}
